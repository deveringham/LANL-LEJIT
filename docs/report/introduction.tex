% 
\section{Introduction}
\label{sec:introduction}

\textbf{Workflow} for scientific computing (collection, generation and processing of numerical data) is 
defined as “sequencing and orchestrating operations, along with the attendant tasks related 
to these operations”\cite{ascr:FutSciWork:2015}.  For example, workflow can include how the problem is
set up and specified, how processing is specified, how control flows between various components
of the application, how data moves between processing stages, and saving some of the workflow
history (called \textbf{provenance}), so that experiments can be explained or even reproduced.  
A significant portion of workflow research also involves how to measure or characterize workloads and
workflow performance.  
This report focuses on investigating ways in which we might make our workflows more flexible
and useful for future Next Generation Code applications.  It does not focus on measuring and 
characterizing performance, but more on exploring how we specify, configure and control workflows
in order to allow for flexibility in the execution of the workflow and ultimately more optimal
performance and quality of solution.  For the purposes of this document, we will call this the 
control workflow.  It is a workflow that sits at a higher level than the workflow that exists 
within application components.  By components we mean executables, packages or libraries that 
compose a scientific application.   This report concerns researching and prototyping 
novel mechanisms for control workflows for new scientific application codes currently being developed.  
